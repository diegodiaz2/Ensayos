% 		INSTRUCTIVO PARA CONSTRUIR EL ENSAYO EN LATEX
%					v.1.3
%				elaborado por @rmuriel
%		---------------------------------------------------
		
% Este instructivo se crea para facilitar la comprensión sobre la estructura de un trabajo escrito en LaTeX, y de paso para que se comprenda una forma (entre las miles que existen) en que se podría construir un «ensayo de ficción».

% Antes de empezar, para algunos será útil que lo diga, se observará que el documento está dividido en títulos en mayúscula y comentarios en color azul (como este). Corresponden a las partes habituales de un documento LaTeX y mis explicaciones. La parte que van a modificar se llama «CUERPO DEL DOCUMENTO». No tienen que saber código LaTeX para hacer este trabajo, sólo se dejan guiar por esta plantilla y listo. Más allá de leer juiciosamente este instructivo, no necesitan más!! :)

% Lo que esté en latín y en color negro es lo que ustedes van a modificar. Verán que es muy fácil... abran la mente y déjense guiar! :)

% Comencemos!! 

%================================================================»
% 0 - RAZONES POR LAS QUE HACEMOS ESTE TRABAJO EN LATEX
%================================================================»

% 1. LaTeX es el lenguaje del mundo académico. Las mejores universidades del mundo funcionan así, en todas existe al menos una cátedra permanente de LaTeX. En Colombia, lastimosamente, sólo lo usa la Universidad Nacional, la Universidad de Antioquia y la Universidad de los Andes. Cada una de estas universidades tiene formatos (plantillas LaTeX) para trabajos de semestre, tesis, informes de laboratorio, ensayos y exámenes parciales. 

% 2. En dos aspectos importantes es el procesador de texto más efectivo que existe: rendimiento de la máquina (no hace que se cuelgue) y acabado editorial del documento (La forma final del documento es profesional y responde a los cánones editoriales internacionales).

% 3. Es software libre y multiplataforma. Se puede trabajar desde casi cualquier sistema operativo respetable. No necesita ser pirateado y está disponible para su descarga las 24 horas del día. Sin contar con que existe https://www.writelatex.com, que hace el trabajo mucho más cómodo de manera online. 

% 4. Automatiza procedimientos mecánicos típicos en la construcción de documentos:  autonumeración de fórmulas, generación de listas, creación de índices de contenido, de tablas, figuras y terminológicos, etc. 

% 5. La preocupación por la forma se la deja uno al computador. Uno no tiene que pensar en las márgenes, en las negritas de los títulos, en el tipo de letra, etc... De esas formalidades se encarga LaTeX... Uno sólo se concentra en producir contenido, que es para eso que se inventaron los procesadores de texto en una computadora. Preocuparse por la forma es como si no se hubiera superado la época de la máquina de escribir. 

% 6. Permite el uso de bases de datos bibliográficas con BibTeX. Se ahorra tiempo a la hora de citar textos y hacer listados de publicaciones. Basta con hacer una vez la base bibliográfica y uno sólo debe «llamar» las referencias usadas para cualquier cantidad de textos que uno escriba. En esto LaTeX está conectado con Mendeley, la base de datos bibliográfica más importante de la actualidad en el mundo académico. 

% y como si esto fuera poco...

% 7. Con LaTeX se permiten hacer comentarios en cualquier sección del texto sin que aparezcan en el documento final. Basta con introducir un signo de porcentaje (%) antes de empezar a escribirlos. 

% De modo que... empecemos desde ya a usar LaTeX!!!!

%================================================================»
% I - PREÁMBULO
%================================================================»

% Antes de escribir el texto como tal, para LaTeX es importante clarificar algunos aspectos básicos sobre la naturaleza del documento que se va a escribir. Esta primera parte se llama «PREÁMBULO» y es el lugar donde se clarifican los siguientes aspectos:

% - Tamaño de hoja y de fuente
% - Tipo de documento: libro, artículo, informe, etc.
% - Paquetes de información: español, márgenes, copiado pdf, colores, gráficos, etc. 
% - Autor, título y fecha
% - Algunos ajustes a la estética del documento general

%----------------------------------------------------------------»
% a - Definición de la Clase del Documento 
%----------------------------------------------------------------»

\documentclass[11pt,letterpaper]{article}

%----------------------------------------------------------------»
% b - Paquetes para trabajar en español 
%----------------------------------------------------------------»

\usepackage[utf8]{inputenc}
\usepackage[spanish]{babel}
\usepackage{babelbib}
\usepackage{url}

%----------------------------------------------------------------»
% c - Paquetes para solucionar el copiado del pdf
%----------------------------------------------------------------»

\usepackage{times}		
\usepackage[T1]{fontenc}	

%----------------------------------------------------------------»
% d - Paquetes especiales (Según las necesidades del documento)
%----------------------------------------------------------------»

\usepackage[colorinlistoftodos]{todonotes} %Para insertar notas al lado
\usepackage{graphicx} %Para usar imágenes
\usepackage{tikz} %Para construir gráficos con código
\usepackage{epigraph} %Hacer epígrafes
\usepackage{multicol} %Construir múltiples columnas en el documento
\usepackage{color} %Para darle color a la fuentes
\usepackage{soul} %Para tachar palabras
\usepackage{ulem} %Para subrayados y tachados especiales (\uuline, \uwave, \xout) Aunque casi nunca se usan, a veces pueden introducirse para remarcar algo. 

%----------------------------------------------------------------»
% e - Paquete para generar links (Si el doc. tiene hipervínculos)
%----------------------------------------------------------------»

\usepackage[backref]{hyperref}	% Soporte para generación de Links - Ojalá siempre el último paquete nombrado
\hypersetup{pdfborder={0 0 0}}	% Quitarle los bordes a los links

%----------------------------------------------------------------»
% f - Arreglos sobre la estética de los párrafos (Opcional)
%----------------------------------------------------------------»

\setlength\parindent{0pt}	% Si se quiere suprimir la sangría de los párrafos
\setlength{\parskip}{2mm}	% Si se quiere espaciar todos los párrafos

%----------------------------------------------------------------»
% g - Autor, título y fecha del Documento
%----------------------------------------------------------------»

\author{DIEGO FERNANDO DÍAZ TORRES\thanks{INGENIERIA ELECTRONICA, UNIVERSIDAD DE ANTIOQUIA, 2020}}
\title{INTERRUPCIONES EN MICROPROCESADORES}
\date{\today} 

%================================================================»
% II - CUERPO DEL DOCUMENTO
%================================================================»

% Después de todo el preámbulo nos adentramos en la escritura del trabajo. El CUERPO DEL DOCUMENTO en LaTeX siempre inicia con las siguientes dos instrucciones:

\begin{document}
\maketitle

% En el CUERPO DEL DOCUMENTO es donde vamos a encontrar:

% - Abstract
% - Secciones y subsecciones
% - Tabla de contenido
% - Tablas
% - Gráficos
% - Notas al pie y al márgen
% - Párrafos especiales (cita)
% - Bibliografía

%----------------------------------------------------------------»
% a - Creación del resumen (Abstract)
%----------------------------------------------------------------»

% El abstract es el resumen del ensayo. Se expone, entre cuatro y siete líneas, la naturaleza del escrito, su tema, el tipo de indagación y los intereses del texto. 


%----------------------------------------------------------------»
% b - Escribir el Epígrafe (Opcional)
%----------------------------------------------------------------»

% Uno puede escribir o no un epígrafe al principio de un ensayo. Ustedes quizá lo han visto con frecuencia en diferentes tipos de escritos (ensayos, novelas, etc.) - Lo importante es que el epígrafe aluda a algo importante que usted quiere comunicar en el ensayo. 


%----------------------------------------------------------------»
% c - Inicio de las secciones del documento
%----------------------------------------------------------------»

\section*{¿Qué es una interrupción?} % La instrucción  \section con el signo * hace que no quede numerado.


% En la «Introducción» se escribe una preparación a la discertación. La idea es atrapar al lector con sus propios intereses. Hacerle caer en cuenta que a él le gustaría leer sobre lo que usted le va a contar, especialmente le gustaría saber las razones por las cuales él debería ser un inventor como usted!!

Las interrupciones son muy importantes en el uso de microcontroladores, esto se debe a que permiten ejecutar un subproceso en un momento exacto, y después de este ser ejecutado volver al programa principal para terminar de ser ejecutado. Estas permiten una mayor efectividad en el uso de tiempo y de eventos en los sistemas embebidos.

\section*{Historia de las interrupciones}
Inicialmente se utilizaba un método llamado “polling”, esta consistía en que el propio procesador sondeaba los dispositivos periféricos en determinados intervalos de tiempo, de esta forma se verificaba si se tenía alguna comunicación para él.
El polling es ineficiente dado que se tiene perdidas de tiempo y de recursos al realizar las instrucciones de eventos. Dado que el sondeo se realiza en determinados tiempos se pueden perder eventos, para evitar estos seria necesario hacer que el sondeo se realice con mas frecuencia, pero esto involucraría un mayor consumo del CPU, y además evitaría que el CPU realizara otras tareas al mismo tiempo.

% ----------------------
% La intrucción \underline se usa para subrayar frases. 
% ----------------------

\section*{Tipos de interrupciones}

Las interrupciones se dividen de la siguiente forma:
\subsection{Excepciones}
Estas son clasificadas como un tipo de interrupción que se usa para informar al SO que hay una condición de error, por ejemplo, una división entre 0, o la lectura de un disco que no existe. Cuando una excepción es detectada se trata de solucionar llamando a la función que trata esta interrupción, en caso de que esta no sea posible de solucionar se termina con la ejecución del proceso.
\subsection{Interrupciones en el Hardware}
Las interrupciones de hardware se pueden producir en cualquier momento, sin importar lo que esté haciendo el CPU. Estas son producidas por causas externas del procesador, y por lo general están ligadas a los dispositivos de entrada y salida.
Este tipo de interrupciones permiten mejorar la productividad del procesador, ya que en vez de hacer que el dispositivo realice una espera activa, puede ir atendiendo otro proceso.
\subsection{Interrupciones en el Software}
Son las interrupciones generadas por el mismo programa, usando una función especial del lenguaje, el objetivo de estas es que el procesador realice algún tipo de función y después de ejecutar esta, se continuara con el programa que provoco la interrupción. Las interrupciones de software tienen mayor prioridad que las de hardware y generalmente son usadas para entrada y salida


% ----------------------
% Se van a dar cuenta de que el subrayado con \underline no funciona si lo que quieren es subrayar un párrafo completo, esta instrucción se usa sólo en palabras o frases cortas. 
% ----------------------

% ----------------------
% La instrucción \footnote{} es para hacer pies de página. Como se ve en el resultado en PDF, generan un numerito consecutivo, a la manera habitual de los pies de página de los artículos o libros de ciencia. 
% ----------------------

\section*{Mecanismo de Interrupción}
1.	Se transmite una señal de interrupción al procesador.\\
2.	El procesador finaliza la ejecución de la instrucción que estaba en proceso.\\
3.	Se envía una señal de reconocimiento al dispositivo que genero la interrupción, y se suprime la señal de esta.\\
4.	Se carga el contador del programa con la ubicación de la entrada al tratamiento.\\
5.	Se procede a realizar la interrupción.\\
6.	Una vez finalizada se restauran los valores salvados antes de realizar la interrupción.\\
7.	El programa continua en la instrucción siguiente a la que genero la interrupción.

\section*{Ejemplo en arduino}
Link en Tinkercad: \url{https://www.tinkercad.com/things/7zZHyUgtJPb}
\subsection*{Codigo fuente:}
//El led se encuentra conectado al pin 8\\
const int pinled = 8;\\
//El pulsador se encuentra conectado al pin 2\\
const int pinpulsador = 2;\\
//Esta es la variable que cambia el estado del led(Apagado o encendido)\\
//Cada vez que se presione este valor varia entre 0 y 1\\
volatile int valor = 0;\\
void setup() {\\
  //Se inicializa el pin como salida\\
   pinMode(pinled, OUTPUT);\\
  //Esta funcion detecta un cambio de alto a bajo\\
   attachInterrupt(digitalPinToInterrupt(pinpulsador), estado, RISING);\\
}\\
void loop() {\\
   //Si valor=0 el led permanece apagado\\
  //Si valor=1 el led permanece encendido\\
   digitalWrite(pinled, valor);\\
}\\
//ISR del pin 2, es decir la funcion que\\
void estado() {\\
  //Cada vez que entre valor cambia\\
   valor = !valor;\\
}


%----------------------------------------------------------------»
% c - Bibliografía
\begin{thebibliography}{a}
\bibitem{a} \textsc{Reyes, F.},
\textit{Arduino. Aplicaciones en robotica y mecatronica}
\url{http://www3.fi.mdp.edu.ar/electrica/opt_archivos/arduino/Manejo_de_Interrupciones.pdf}
\bibitem{b} \textsc{Domínguez, R.},
\textit{Interrupciones}
\url{https://raulalejandroql.webcindario.com/atmel/8_Interrupciones.pdf}
\bibitem{c} 
\textit{Estructura de Computadores. Facultad de Informática}
\url{http://www.fdi.ucm.es/profesor/jjruz/WEB2/Temas/Curso05_06/EC9.pdf}
\end{thebibliography}
%----------------------------------------------------------------»

% El entorno «thebibliography» nos sirve para construir la bibliografía. Cada \bibtem es una referencia que hemos usado en nuestro documento. 

% Para citar las referencias usamos el comando \cite{etiqueta}, tal y como se hizo en el último párrafo de esta plantilla.  Por supuesto, la «etiqueta» es el nombre que le hemos dado a la referencia. En el caso del primer libro de esta bibliografía vemos que la etiqueta es «ejemplo», las otras son «libro1», «libro2», etc. Usted puede usar cualquier etiqueta siempre y cuando no se repita en otra referencia. Cada referencia tiene etiqueta única.


%================================================================»
% EXPLICACIONES FINALES
%================================================================»
%----------------------------------------------------------------»
% Signos en LaTeX
%----------------------------------------------------------------»

% Como se ha notado, escribir el signo % (porcentaje) produce «comentarios» dentro del código, explicaciones que no son tomadas en cuenta a la hora de «compilar» el código escrito. Si se quiere incorporar un signo % (porcentaje) como parte del texto que se está escribiendo debe escribirse con la barra de instrucción habitual, así: \% 

% Hay otros signos a los que también es necesario antecederlos de la barra \ - Son los siguientes:

% \		carácter inicial de comando			Se escribiría: \tt\char‘\\
% { }	abre y cierra bloque de código		Se escribiría: \{, \}
% $		abre y cierra el modo matemático		Se escribiría: \$
% &		tabulador (en tablas y matrices)		Se escribiría: \&
% #		señala parámetro en las macros		Se escribiría: \_ , \^{}
% _, ^	para subíndices y exponentes			Se escribiría: \#
% ~		para evitar cortes de renglón			Se escribiría: \~{}

%----------------------------------------------------------------»
% Cambios en la estética de las palabras
%----------------------------------------------------------------»

% Este es el listado de las instrucciones básicas:

% - Negrita: 	\textbf{}
% - Itálica: 	\textit{}
% - Slanted:		\textsl{}
% - Sans Serif:	\textsf{}
% - Versalitas:	\textsc{}
% - Typewriter: 	\texttt{}
% - Enfático:	\emph{}

% Lo que se escriba dentro de los corchetes de cada instrucción será lo que se verá modificado en el texto. Ejemplo:

% \sc{Esto es una frase en versalitas}

% Por supuesto, la anterior instrucción no compilará en este documento porque la antece un signo de % (porcentaje), que es el signo de los «comentarios». Pero, pruebe en el texto normal y verá los cambios con cada una de las anteriores instrucciones. 

%--------------------------------»»
% NOTA IMPORTANTE
% Si alguien quiere anexar tablas o gráficos al documento, le recomiendo acercarse a la sección que lo explica en los manuales, guías o instructivos que están en BlackBoard. 
%--------------------------------»»

\end{document}